\documentclass[preprint,12pt]{elsarticle}

%% Use the option review to obtain double line spacing
%% \documentclass[preprint,review,12pt]{elsarticle}

%% Use the options 1p,twocolumn; 3p; 3p,twocolumn; 5p; or 5p,twocolumn
%% for a journal layout:
%% \documentclass[final,1p,times]{elsarticle}
%% \documentclass[final,1p,times,twocolumn]{elsarticle}
%% \documentclass[final,3p,times]{elsarticle}
%% \documentclass[final,3p,times,twocolumn]{elsarticle}
%% \documentclass[final,5p,times]{elsarticle}
%% \documentclass[final,5p,times,twocolumn]{elsarticle}

%% The graphicx package provides the includegraphics command.
\usepackage{graphicx}
\usepackage{hyperref}
%% The amssymb package provides various useful mathematical symbols
\usepackage{amssymb}

\usepackage{amsmath}

%% The lineno packages adds line numbers. Start line numbering with
%% \begin{linenumbers}, end it with \end{linenumbers}. Or switch it on
%% for the whole article with \linenumbers after \end{frontmatter}.
\usepackage{lineno}

\journal{bioRxiv}

%% natbib.sty is loaded by default. However, natbib options can be
%% provided with \biboptions{...} command. Following options are
%% valid:

%%   round  -  round parentheses are used (default)
%%   square -  square brackets are used   [option]
%%   curly  -  curly braces are used      {option}
%%   angle  -  angle brackets are used    <option>
%%   semicolon  -  multiple citations separated by semi-colon
%%   colon  - same as semicolon, an earlier confusion
%%   comma  -  separated by comma
%%   numbers-  selects numerical citations
%%   super  -  numerical citations as superscripts
%%   sort   -  sorts multiple citations according to order in ref. list
%%   sort&compress   -  like sort, but also compresses numerical citations
%%   compress - compresses without sorting
%%
%% \biboptions{comma,round}

% \biboptions{}

\begin{document}

\begin{frontmatter}

%% Title, authors and addresses

\title{Why you may want to model DNA replication in stochastic models of synthetic gene circuits (and how)}

%% use the tnoteref command within \title for footnotes;
%% use the tnotetext command for the associated footnote;
%% use the fnref command within \author or \address for footnotes;
%% use the fntext command for the associated footnote;
%% use the corref command within \author for corresponding author footnotes;
%% use the cortext command for the associated footnote;
%% use the ead command for the email address,
%% and the form \ead[url] for the home page:
%%
%% \title{Title\tnoteref{label1}}
%% \tnotetext[label1]{}
%% \author{Name\corref{cor1}\fnref{label2}}
%% \ead{email address}
%% \ead[url]{home page}
%% \fntext[label2]{}
%% \cortext[cor1]{}
%% \address{Address\fnref{label3}}
%% \fntext[label3]{}


%% use optional labels to link authors explicitly to addresses:
%% \author[label1,label2]{<author name>}
%% \address[label1]{<address>}
%% \address[label2]{<address>}

\author{Samuel Clamons}
\author{Richard Murray}

\address{Caltech, Pasadena, CA, United States}


\begin{abstract}
NO ABSTRACT YET.
\end{abstract}

\end{frontmatter}

%% main text
\section{Suggested Models for Replicating DNA}

TABLE:
\begin{enumerate}
\item Model name
\item CRN description
\item English description
\item Diagrammatic explanation
\item Steady-state distributions fit against Voigt's data
\item Relative simulation speed (plasmid only)
\end{enumerate}

% MODEL DETAILS
\section{Model Details}

\subsection{What is this?}

\subsection{Why should I care?}

	See \hyperref[S:motivation]{Section \ref{S:motivation}}

\subsection{What in the world is that ``Dummy Molecule'' in the first model? What kind of molecule is it supposed to represent?}

\subsection{There's a lot going on in the Brendel \& Perelson model. What are all of those states?}

\subsection{What's the relationship between the ``Brendel \& Perelson'' and ``Reduced Brendel \& Perelson'' models?}

% EVALUATING MODELS
\section{Evaluating Models}

\subsection{What's with those ``steady-state distributions''?}

\subsection{What are the ``empirical distributions'' you're showing for copy number?} 

\subsection{How did you fit the models against the empirical distributions?}

\subsubsection{Dummy replication}

\subsubsection{Brendel \& Perelson, Reduced Brendel \& Perelson}

\subsection{So these models will hold plasmids at constant copy number ($\pm$ noise)?}

\subsection{Which model should I use? Why?}

% MOTIVATION
\section{Motivation}\label{S:motivation}

\subsection{Why did you write this?}

\subsection{I've never had to model DNA replication in my models. Why would I ever need to?}

\subsection{Can't you just...?}

\subsection{So these models are only useful for stochastic simulations?}

\subsubsection{Then if I'm using deterministic ODEs to model my circuit, I can just ignore DNA replication, right?}

\subsection{When would you \emph{actually need} to model DNA replication? Give me a concrete example.}

	(CRISPRi oscillator)

\subsubsection{Is there \emph{really} no other way to model this circuit?}

\subsection{I'm not convinced these are necessary. Can you give another concrete example?}

	(temporal logic gate)
	
\subsubsection{Is there \emph{really} no other way to model this circuit?}

% DATA AVAILABILITY AND REPRODUCIBILITY
\section{Data/Code Availability and Reproducibility}

\subsection{What software did you use to simulate this stuff?}

Python. Python packages. BioSCRAPE. Whatever that tool was to scrape distributions from Voigt's paper.

Versions of everything! 

\subsection{Do you have code for any of this? Where can I get it?}

Mention that you can set up a python env using code in repo, or just run it if you're confident you have everything you need.

\subsection{What's the timing information about, and why is it there? What should I use those numbers for?}

\subsection{What machine did you run on to get timing information?}

% Figure example

%\begin{figure}[h]
%\centering
%\includegraphics[scale=1]{activators_vs_repressors.pdf}
%\caption{Changes in expression when the concentrations of two regulators (an activator and a repressor) are changed under different regulator concentration regimes. Both the repressor and the activator change expression 10-fold (top table). At low saturation of target promoter by the regulators, the activated promoter is more sensitive to changes in regulator concentration than the regulated promoter (middle table). Conversely, at high target saturation, the activated promoter is more robust to changes in regulator concentration (bottom table).}
%\label{actvsrep}
%\end{figure}


%% References with bibTeX database:
\section{References}

%\bibliographystyle{model1-num-names}
\bibliographystyle{plain}
\bibliography{plasmid_replication.bib}

%% Authors are advised to submit their bibtex database files. They are
%% requested to list a bibtex style file in the manuscript if they do
%% not want to use model1-num-names.bst.

%% References without bibTeX database:

% \begin{thebibliography}{00}

%% \bibitem must have the following form:
%%   \bibitem{key}...
%%

% \bibitem{}

% \end{thebibliography}


\end{document}

%%
%% End of file `elsarticle-template-1-num.tex'.